\documentclass[a4paper]{article}
\usepackage[OT2]{fontenc}
\usepackage[english,russian]{babel}
\usepackage{cjhebrew}
\begin{document}
\def\latin#1{\cjLR{\rmfamily\normalsize#1}}
\Large

\section*{\foreignlanguage{russian}{Неделя 2 урок 1}}
\subsection*{\foreignlanguage{russian}{Буквы}}
\begin{cjhebrew}

h \latin{\foreignlanguage{russian}{(Х незвучащее) --- Хе} ---}

.h \latin{\foreignlanguage{russian}{(Х) --- Хет} ---}

\end{cjhebrew}

\subsection*{\foreignlanguage{russian}{Слова}}
\begin{cjhebrew}

.hag \latin{\foreignlanguage{russian}{(Праздник) --- Хаг} ---}

ha.hag \latin{\foreignlanguage{russian}{(Празднество) --- Ах\'{а}г} ---}

hadag \latin{\foreignlanguage{russian}{(Рыбища) --- Ад\'{а}г} ---}

\end{cjhebrew}

\section*{\foreignlanguage{russian}{Неделя 2 урок 2}}
\subsection*{\foreignlanguage{russian}{Буквы}}
\begin{cjhebrew}

n\dottedcircle~ \latin{\foreignlanguage{russian}{(Н) --- Нун} ---}

N \latin{\foreignlanguage{russian}{(Н в конце слова) --- Нун} ---}

~E \latin{\foreignlanguage{russian}{(E) --- Сег\'{о}л} ---}

~e \latin{\foreignlanguage{russian}{(E) --- Цер\'{е}} ---}

\end{cjhebrew}

\subsection*{\foreignlanguage{russian}{Слова}}
\begin{cjhebrew}

b*en \latin{\foreignlanguage{russian}{(Сын) --- Бен} ---}

.hanAh \latin{\foreignlanguage{russian}{(Ханна) --- Х\'{а}на} ---}

gan \latin{\foreignlanguage{russian}{(Сад) --- Ган} ---}

ganAn \latin{\foreignlanguage{russian}{(Садовник) --- Ганан} ---}

hEgEh \latin{\foreignlanguage{russian}{(Руль) --- Еге} ---}

bAnAnAh \latin{\foreignlanguage{russian}{(Банан) --- Банана} ---}

\end{cjhebrew}
\end{document}
